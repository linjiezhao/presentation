%%%%%%%%%%%%%%%%%%%%%%%%%%%%%%%%%%%%%%%%%
% a0poster Portrait Poster
% LaTeX Template
% Version 1.0 (22/06/13)
%
% The a0poster class was created by:
% Gerlinde Kettl and Matthias Weiser (tex@kettl.de)
%
% This template has been downloaded from:
% http://www.LaTeXTemplates.com
%
% License:
% CC BY-NC-SA 3.0 (http://creativecommons.org/licenses/by-nc-sa/3.0/)
%
%%%%%%%%%%%%%%%%%%%%%%%%%%%%%%%%%%%%%%%%%

%----------------------------------------------------------------------------------------
%	PACKAGES AND OTHER DOCUMENT CONFIGURATIONS
%----------------------------------------------------------------------------------------

\documentclass[a0,portrait]{a0poster}

\usepackage{multicol} % This is so we can have multiple columns of text side-by-side
\columnsep=100pt % This is the amount of white space between the columns in the poster
\columnseprule=3pt % This is the thickness of the black line between the columns in the poster

\usepackage[svgnames]{xcolor} % Specify colors by their 'svgnames', for a full list of all colors available see here: http://www.latextemplates.com/svgnames-colors

\usepackage{times} % Use the times font
%\usepackage{palatino} % Uncomment to use the Palatino font

\usepackage{graphicx} % Required for including images
\graphicspath{{figures/}} % Location of the graphics files
\usepackage{booktabs} % Top and bottom rules for table
\usepackage[font=small,labelfont=bf]{caption} % Required for specifying captions to tables and figures
\usepackage{amsfonts, amsmath, amsthm, amssymb} % For math fonts, symbols and environments
\usepackage{wrapfig} % Allows wrapping text around tables and figures
\usepackage{color} %设置背景颜色宏包
\usepackage{enumerate}
\usepackage{mathrsfs}
\usepackage{float}
\usepackage{cite}

%\definecolor{myback}{RGB}{204,232,207}%定义背景颜色
%\pagecolor{myback}%设置背景颜色

\newtheorem*{theorem}{Theorem}


\begin{document}



%----------------------------------------------------------------------------------------
%	POSTER HEADER
%----------------------------------------------------------------------------------------

% The header is divided into two boxes:
% The first is 80% wide and houses the title, subtitle, names, university/organization and contact information
% The second is 20% wide and houses a logo for your university/organization or a photo of you
% The widths of these boxes can be easily edited to accommodate your content as you see fit

\begin{minipage}[b]{0.8\linewidth}
\Huge \color{NavyBlue} \textbf{Hydrodynamics of the Binary Contact Path Process} \color{Black}\\ % Title
\begin{center}
\huge {Xiaofeng Xue \footnote{\large {\textbf{E-mail}: xfxue@bjtu.edu.cn \textbf{Address}: School of Science, Beijing Jiaotong University, Beijing 100044, China.}} and {\bf Linjie Zhao} \footnote{\large {\textbf{E-mail}: zhaolinjie@pku.edu.cn \textbf{Address}: School of Mathematical Sciences, Peking University, Beijing 100871, China.}}}\\[0.5cm] % Author(s)
\end{center}
\end{minipage}
%
\begin{minipage}[b]{0.2\linewidth}
\begin{center}
\includegraphics[width=15cm]{zhao.jpg}\\
\end{center}
\end{minipage}

\vspace{1cm} % A bit of extra whitespace between the header and poster content

%----------------------------------------------------------------------------------------

\begin{multicols}{2} % This is how many columns your poster will be broken into, a portrait poster is generally split into 2 columns

\color{DarkSlateGray} % DarkSlateGray color for the rest of the content



\section*{Contact Process}

\begin{itemize}
  \item Introduced by Harris \cite{Har1974} to describe the spread of a  disease on $\mathbb{Z}^d$.
  \item A popular model in the area of interacting particle systems (see Liggett's two monographs \cite{Lig1985,liggett99}).
  \item Appeared independently in the high-energy physics literature, and is equivalent to the reggeon spin model.
  \item Simple but exhibits a phase transition.
\end{itemize}


\vskip 0.2cm

Regard each $x \in \mathbb{Z}^d$ as an individual. The state $\eta (x)$ of $x$ takes value in $\{0,1\}$. The individual $x$ is healthy if $\eta(x)=0$, while is infected if $\eta(x) = 1$.  The dynamics is as follows:
\begin{enumerate}
\item Each infected individual is recovered at rate $1$, i.e., $1 \rightarrow 0$ at rate $1$.
\item An individual $x$ is infected by an infected neighbor $y$ at rate $\lambda$, i.e., $0 \rightarrow 1$ at rate $\lambda \sum_{y \sim x} \eta (y)$.
\end{enumerate}

\vskip 0.2cm

{\bf \color{SaddleBrown} Critical Value.} It is easy to see that the probability that the disease survives is increasing as $\lambda$ increases. Therefore, there exists a critical value $\lambda_c$ such that
\begin{equation*}
  \text{The disease}
  \begin{cases}
    \text{is extinct with probability one}, & \mbox{if } \lambda < \lambda_c, \\
    \text{survives with positive probability}, & \mbox{if} \lambda > \lambda_c.
  \end{cases}
\end{equation*}


\section*{Binary Contact Path Process}

On top of the contact process, we also consider the {\it seriousness} of the disease. The state $\eta (x)$ of $x$ takes value in $[0,\infty)$. The individual $x$ is healthy if $\eta(x)=0$, while is infected if $\eta(x)>0$ and the value of $\eta(x)$ is the seriousness of the disease on $x$. The dynamics is as follows:
\begin{enumerate}[(I)]
\item Each infected individual is recovered at rate $1$.
\item An individual $x$ is infected by a given neighbor $y$ at rate $\lambda$. When the infection occurs, the seriousness of the disease on $x$ is added  with that of $y$.
\item  When there is no infection occurs for $x$ during some time interval, then $\eta_t (x)$ evolves according to the deterministic ODE
    $$
    \frac{d}{dt}\eta_t(x)=(1-2\lambda d)\eta_t(x).
    $$
\end{enumerate}

\vskip 0.2cm

\begin{itemize}
  \item Introduced by Griffeath \cite{Gri1983} as  an auxiliary model to study the critical value of the contact process: If $d \geq 3$, then
      \begin{equation*}
        \lambda_c \leq \frac{1}{2d(2\gamma_d-1)},
      \end{equation*}
  where $\gamma_d$ is the probability that the simple random walk on $\mathbb{Z}^d$ starting at $O$ never return to $O$ again.
  \item Let
  \[
\xi_t(x)=
\begin{cases}
1 &\text{~if~}\eta_t(x)>0,\\
0 &\text{~if~}\eta_t(x)=0,
\end{cases}
\]
for each $x\in \mathbb{Z}^d$, then $\{ \xi_t\}$ is a version of the contact process.
\end{itemize}

\section*{Hydrodynamics}

The theory says that the microscopic density field of the concerned model, after properly space time scaling, is dominated macroscopically  be some PDE ({\it hydrodynamic equation}). We refer the reader to \cite{kipnis+landim99}.

\vskip 0.2cm

For each $N \geq 1$, let $\{\eta^N_t\}$ be the binary contact path process with {\bf \color{SaddleBrown} time speeded up by $N^2$}. We are concerned about the random empirical measure $\pi^N_t$ of the process defined by
\[
\pi^N_t := \frac{1}{N^d}\sum_{x\in \mathbb{Z}^d}\eta_t^N(x)\delta_{\frac{x}{N}}(du).
\]

\vskip 0.2cm

{\color{SaddleBrown}
\begin{center}
\begin{tabular}[b]{|c|c|c|}
\hline
      & Macroscopic & Microscopic          \\ \hline
Time  & $t$           & $t N^2$ \\ \hline
Space & $x / N$           & $x$                      \\ \hline
Space & $\mathbb{R}^d$           & $\mathbb{Z}^d$                      \\ \hline
\end{tabular}
\end{center}
}


{



\section*{Main Result}

We proved the law of large numbers for the empirical measure $\pi^N_t$. The hydrodynamic equation turns out to be the heat equation.

\color{SaddleBrown}

\begin{theorem}
~~Let $\rho_0: \mathbb{R}^d \rightarrow [0,\infty)$ be bounded and integrable. Initially, $\eta^N_0 (x) = \rho_0 (x / N)$. Suppose $d \geq 3$ and
\[
\lambda>\frac{1}{2d(2\gamma_d-1)},
\]
then for all $t \geq 0$, as $N \rightarrow \infty$,
\begin{equation}\label{}
  \pi^N_t (d u) \rightarrow \rho (t, u) du~~\text{in probability,}
\end{equation}
where $\rho (t,u)$ is the unique solution of the heat equation
\begin{equation}\label{eq:intro1}
  \begin{cases}
    \partial_t \rho (t,u) = \lambda \Delta \rho (t,u), \\
    \rho (0,u) = \rho_0 (u).
  \end{cases}
\end{equation}
\end{theorem}


}

\section*{Remarks}

\begin{itemize}
  \item The favorite models in hydrodynamic theory are generally mass conserved, such as exclusion processes and zero range processes. However, the binary contact path process lacks this property.
  \item We are able to calculate the hydrodynamics of the binary contact path process due to the following two facts:
      \begin{itemize}
        \item { \color{SaddleBrown} On average, the total mass of the system is conserved, see Equation \eqref{eq2}.} This fact is due to the dynamics (III) of the process.
        \item The binary contact path process belongs to { \color{SaddleBrown} linear systems}, which allows explicit calculations.
      \end{itemize}
  \item  A central limit theorem for the density of particles in the context of binary context path processes was proved in \cite{NagahataYoshida09}.
  \item It remains open to study the hydrodynamics of the process for small $\lambda$ or in lower dimensions $d \leq 2$.
\end{itemize}

\section*{Sketch of the Proof}

The proof is divided into four steps:
\begin{enumerate}[Step 1.]
  \item Let $Q^N$ be the law of $\{ \pi^N_t, 0 \leq t \leq T\}$. Then $\{Q^N, N \geq 1\}$ is tight. { ( \color{SaddleBrown} Prohorov's Theorem and Aldous' Criterion.)}
  \item Let $Q^*$ be any limit of $Q^N$. Then $Q^*$ is concentrated on absolutely continuous trajectories.
  \item Show that the density of the trajectory is the solution of the heat equation \eqref{eq:intro1} with probability one with respect to $Q^*$. { ( \color{SaddleBrown} Martingale Approach.)}
  \item The result follows from the uniqueness of the solution of the heat equation.
\end{enumerate}

\vskip 0.3cm

The {\bf \color{SaddleBrown} difficulty} is to prove the {\bf \color{SaddleBrown} absolute continuity (Step 2)} since the value at each site is not bounded. We overcome this difficulty by proving the convergence to zero of the variance: given $G\in C_c^2(\mathbb{R}^d)$ and $t>0$,
\begin{equation}\label{eq1}
  \lim_{N\rightarrow+\infty}{\rm Var}\big( \langle \pi_t^N,G \rangle \big)=0.
\end{equation}
This implies that
\begin{equation*}
  \langle \pi_t,G \rangle = \mathbf{E}_{Q^*} \left[ \langle \pi_t,G \rangle \right] ~~~\text{$Q^*$-a.s.}
\end{equation*}
Then (B) follows from the following fact:
\begin{equation}\label{eq2}
   \mathbb{E} \left[ \langle \pi_t^N,G \rangle \right] = \frac{1}{N^d} \sum_{x \in \mathbb{Z}^d} \rho_0 \left(\frac{x}{N}\right) G \left(\frac{x}{N}\right),
\end{equation}

\vskip 0.3cm

The proof of Equation \eqref{eq1} is based on {\it direct calculation}. We first write ${\rm Var}\big( \langle \pi_t^N,G \rangle \big)$ as
$$
\frac{1}{N^{2d}}\sum_{x\in \mathbb{Z}^d} \sum_{y\in  \mathbb{Z}^d} G\left(\frac{x}{N}\right) G\left(\frac{y}{N}\right) \Big(\mathbb{E}\big(\eta_t^N(x)\eta_t^N(y)\big)- \mathbb{E}\big(\eta_t^N(x)\big)\mathbb{E}\big(\eta_t^N(y)\big)\Big).
$$
By  Hille-Yosida Theorem and infinitesimal generator calculations,
\[
\mathbb{E}\big(\eta_t(x)\big)=\sum_{y\in \mathbb{Z}^d}p_{t}(x,y)\mathbb{E}\big(\eta_0(y)\big),
\]
where $\{ p_t (x,y) \}$ is the transition probability of the continuous time simple random walk on $\mathbb{Z}^d$ with rate $2 d \lambda$, and there exists a $(\mathbb{Z}^d\times \mathbb{Z}^d)\times(\mathbb{Z}^d\times \mathbb{Z}^d)$ matrix $M_\lambda$ such that
\[
\mathbb{E}\Big(\eta_t(x)\eta_t(y)\Big)=\sum_{u\in \mathbb{Z}^d}\sum_{v\in \mathbb{Z}^d}e^{t M_\lambda}\Big((x,y),(u,v)\Big)\mathbb{E}\Big[\eta_0(u)\eta_0(v)\Big].
\]
Finally, Equation \eqref{eq1} follows from the following result: For any $\lambda>\frac{1}{2d(2\gamma_d-1)}$, there exists $h_\lambda>0$ such that
\[
\sum_{u\in \mathbb{Z}^d}\sum_{v\in \mathbb{Z}^d}e^{tM_\lambda}\big((x,y),(u,v)\big)\leq \frac{k(y-x)+h_\lambda}{h_\lambda},
\]
where $k (x)$ is the probability that the simple random walk hits site $x$ finally when starting from the origin.

\begin{thebibliography}{99}
\bibitem{DeMasiPresutti91} De Masi, A. and Presutti, E. (1991). Mathematical methods for hydrodynamic limits. {\it Lecture Notes in Mathematics}.
\bibitem{Deimling1977}Deimling, K. (1977). \emph{Ordinary Differential Equations in Banach Spaces.} Springer, Berlin.
\bibitem{Dobrushin&Siegmund-Schultze82} Dobrushin Moscow, R. O. and Siegmund-Schultze, R. (1982). The hydrodynamic limit for systems of particles with independent evolution. {\it Mathematische Nachrichten}, {\bf 105}, 199-224.
\bibitem{GalvesKipnisMarchioroPresutti81} Galves, A., Kipnis, C., Marchioro, C. and Presutti, E. (1981). Nonequilibrium measures which exhibit a temperature gradient: study of a model. {\it Communications in Mathematical Physics}, {\bf 81}, 127-147.
\bibitem{Gri1983}Griffeath, D. (1983). The binary contact path process. \emph{The Annals
of Probability} \textbf{11}, 692-705.
\bibitem{Har1974}Harris, T. E. (1974). Contact interactions on a lattice. \emph{The Annals of Probability} \textbf{2}, 969-988.
\bibitem{kipnis+landim99} Kipnis, C. and Landim, C. (1999) {\it Scaling limits of interacting particle systems.} Springer-Verlag, Berlin.
\bibitem{Lig1985}Liggett, T. M. (1985). {\it Interacting Particle Systems.} Springer, New York.
\bibitem{liggett99} Liggett, T. M. (2013). {\it Stochastic interacting systems: contact, voter and exclusion processes} (Vol. 324). springer science $\&$ Business Media.
\bibitem{NagahataYoshida09} Nagahata, Y. and Yoshida, N. (2009). Central limit theorem for a class of linear systems. {\it Electronic Journal of Probability}, {\bf 14}, 960-977.
\bibitem{PresuttiSpohn83} Presutti, E. and Spohn, H. (1983). Hydrodynamics of the voter model. {\it The Annals of Probability}, 867-875.
\bibitem{Rost81} Rost, H. (1981) Nonequilibrium behaviour of a many particle process: density profile and local equilibria. {\it Z. Wahrsch. Verw. Gebiete}, {\bf 58}, 41-53.
\end{thebibliography}


\end{multicols}


\end{document} 